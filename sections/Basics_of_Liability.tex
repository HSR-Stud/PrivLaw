% !TeX spellcheck = en_US
\section{Basics of liability (Haftung)}
\subsection{Damage}
\begin{compactitem}
	\item Damage $\neq$ Satisfaction
	\item Damage = Financial compensation to refund an asset, that has been reduced (or couldn’t rise) by an unlawfully action.
	\item The exact amount has to be proven by submitting invoices, accounting documents etc. The judge has the right to estimate the damage. But he needs evidences.
	\item Satisfaction = Financial compensation for physical or mental injury upon discretion.
\end{compactitem}

\subsection{Multiple \& cumulative requirements for compensation}
\begin{compactenum}
	\item Damage
	\item Unlawfulness
	\item Adequate causal connection
	\item Fault
	\item (no expiry of liability claims)
\end{compactenum}

\subsection{Unlawfulness}
\begin{compactitem}
	\item Includes the proof that a legally protected right has been illegally infringed.
	\item Any infringement of an absolutely protected legal right such as the right to life and limb and property is per se illegal.
	\item Infringement of a relative protected legal right, such as an asset, is only illegal if a standard of protection has been violated.
	\item As an exception and if there are special reasons, the illegality can be suspended. Possible justification grounds include self-defense, consent by the injured party, official obligation etc.
\end{compactitem}

\subsection{Adequate causal connection}
\begin{compactitem}
	\item The proof must be provided that there is a direct causal connection that makes sense in the normal course of life between the damaging event and the damages being claimed.
	\item This means that in the normal course of things and according to general life experience, the damaging	behavior was liable to cause damages of the type that occurred (Definition by the Swiss Federal Court).
\end{compactitem}

\subsection{Fault}
\begin{compactitem}
	\item From an objective viewpoint, fault requires that the damaging party can be judged. From a subjective viewpoint, it requires intent or negligence (Vorsatz oder Fahrlässigkeit). Even minor negligence, ie. simply a violation of due care, is sufficient to constitute fault.
	\item Fault does not need to be proven in the case of contractual liability since it is assumed by law! (see also Art. 97 OR).
	\item Strict liability means that the damaging party is liable irrespective of fault. Liability is founded here on the fact that he is responsible eg. for a generally dangerous circumstance. For example, in animal keeper's liability according to OR 56, proof that the damages were caused by an animal under the care of the animal keeper and that an adequate causal relationship exists is sufficient.
\end{compactitem}

\subsection{Expiry of liability claims}
\begin{compactitem}
	\item Liability claims are subject to a statute of limitations. When the statute of limitations expires, the claims do not go away; they simply can no longer be legally executed against the will of the liable party.
	\item The general statutes of limitations in liability law are:
	\begin{compactitem}
		\item 1 year relative/10 years absolute in non-contractual liability (Art. 60, CO)
		\item 5 or 10 years in contractual liability (Art. 127 ff., CO)
		\item other rules can be found in some special laws
	\end{compactitem}
\end{compactitem}

\subsection{Splitting the damage between several responsible}
\begin{compactitem}
	\item What, when multiple people caused a damage? Then they are regarded as a „Simple Society“.
	\item Externally (damaged party vs. group) they are liable in solidarity („one for all, all for one“)!
	\item Internally (any member of the group) they are liable equally.
\end{compactitem}

\subsection{Contractual and non contractual liability}
\begin{compactitem}
	\item In liability law, a distinction is made between contractual and non-contractual liability. A contractual liability is in effect when an existing contract between the damaging party and the injured party has been violated.
	\item A non-contractual liability is in effect when a person cause a damage, regardless of whether a contract is in place or not.
\end{compactitem}

\subsection{Fault-dependent and fault-independent (causal) liability}
\begin{compactitem}
	\item With regard to non-contractual liabilities, the legislature distinguishes between general fault-dependent liability situations (fault-based liability Art. 41, CO) and fault-independent liability situations listed to some degree in the OR, CC and special laws (causal liability, e.g. in Art. 55, OR, Art. 333, CC).
	\item Product liability is one of the causal independent liability situations and is regulated in a special law, the Product Liability Law (PrHG).
\end{compactitem}

\subsection{Product liability}
\begin{compactitem}
	\item WHAT? Only personal injuries or property damages of consumers are covered (Art. 1, PrHG).
	\item WHO? The manufacturer or the importer are liable (Art. 2, PrHG).
	\item WHAT? Products are movable things (Art. 3 PrHG). No software!
	\item EXCEPTIONS? Manufacturer are not liable eg. if they can prove that they did not bring the product to market or the defect that caused the damage did not exist at the time the product was delivered (Art. 5 PrHG)
	\item WHEN? Claims according to PrHG expire 3 years after the date on which the damaged party becomes aware or should have become aware of the damages, the defect and the manufacturer (Art. 9, PrHG).
\end{compactitem}

\subsection{Legal base}
\subsubsection{Obligation in tort - ART. 41 OR}
Any person who unlawfully causes loss or damage to another, whether willfully or negligently, is obliged to provide compensation.

